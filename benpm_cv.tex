%%%%%%%%%%%%%%%%%%%%%%%%%%%%%%%%%%%%%%%%%%%%%%%%%%%%%%%%%%%%%%%%%%%%%%
% LaTeX Template: Curriculum Vitae
%
% Source: http://www.howtotex.com/
% Feel free to distribute this template, but please keep the
% referal to HowToTeX.com.
% Date: July 2011
%
%%%%%%%%%%%%%%%%%%%%%%%%%%%%%%%%%%%%%%%%%%%%%%%%%%%%%%%%%%%%%%%%%%%%%%
% How to use writeLaTeX:
%
% You edit the source code here on the left, and the preview on the
% right shows you the result within a few seconds.
%
% Bookmark this page and share the URL with your co-authors. They can
% edit at the same time!
%
% You can upload figures, bibliographies, custom classes and
% styles using the files menu.
%
% If you're new to LaTeX, the wikibook is a great place to start:
% http://en.wikibooks.org/wiki/LaTeX
%
%%%%%%%%%%%%%%%%%%%%%%%%%%%%%%%%%%%%%%%%%%%%%%%%%%%%%%%%%%%%%%%%%%%%%%
\documentclass[paper=a4,fontsize=11pt]{scrartcl} % KOMA-article class

\usepackage[english]{babel}
\usepackage[utf8x]{inputenc}
\usepackage[protrusion=true,expansion=true]{microtype}
\usepackage{amsmath,amsfonts,amsthm}     % Math packages
\usepackage{graphicx}                    % Enable pdflatex
\usepackage[svgnames]{xcolor}            % Colors by their 'svgnames'
\usepackage{geometry}
	\textheight=700px                    % Saving trees ;-)
\usepackage{tcolorbox}
\usepackage{multicol}
\usepackage{amssymb}
\usepackage{hyperref}
\usepackage{tabularx}
\usepackage{enumitem}

\pagestyle{empty}           % No pagenumbers/headers/footers

%%% Custom font
%%% ------------------------------------------------------------
\usepackage[sfdefault]{cabin}
\usepackage[T1]{fontenc}

% Adjust margins
\geometry{left=1cm, top=1cm, right=1cm, bottom=1cm}

%%% Custom sectioning (sectsty package)
%%% ------------------------------------------------------------
\usepackage{sectsty}

\sectionfont{%			            % Change font of \section command
	\noindent
	\vspace{-8pt}
	\usefont{OT1}{phv}{b}{n}%		% bch-b-n: CharterBT-Bold font
	\sectionrule{0pt}{0pt}{-4pt}{1pt}}

%%% Colors
%%% ------------------------------------------------------------
% https://coolors.co/fdfffc-235789-c1292e-f1d302-161925
\definecolor{yel}{HTML}{F1D302}
\definecolor{clr_grey}{HTML}{e0e0e0}
\definecolor{clr_darkgrey}{HTML}{606060}
\definecolor{red}{HTML}{C1292E}
\definecolor{blu}{HTML}{235789}
\definecolor{blk}{HTML}{161925}

\hypersetup{
	colorlinks=true
}

\makeatletter
\newcommand{\globalcolor}[1]{%
  \color{#1}\global\let\default@color\current@color
}
\makeatother

\AtBeginDocument{\globalcolor{blk}}

\setlist{nosep}
\setlist[itemize]{leftmargin=*}
\setlist[itemize,1]{label=\textbullet}

\newtcbox{\Gbox}{
	colback=clr_grey,
	colframe=clr_grey,
	nobeforeafter,
	box align=base,
	size=fbox}

%%% Macros
%%% ------------------------------------------------------------
\newlength{\spacebox}
\settowidth{\spacebox}{8888888888}			% Box to align text
\newcommand{\sepspace}{\vspace*{0.5em}}		% Vertical space macro
\newcommand{\showurl}[1]{\href{#1}{#1}}

\newcommand{\MyName}[5]{ % Name
	\noindent
	\tcbset{colback=clr_grey,colframe=clr_grey,boxsep=0mm,on line}
	\tcbox{\huge \usefont{OT1}{phv}{b}{n} #1} \hfill
	\Large \usefont{OT1}{phv}{m}{n} \textit{#2}
	\normalsize \normalfont
}

\newcommand{\NewPart}[1]{
	\vspace{-1em}
	\section*{#1}
}

\newcommand{\SkillsEntry}[2]{
		\noindent
		\begin{tabularx}{\linewidth}{ p{7em} X }
			\noindent
			\begin{minipage}[t]{7em}
				\textbf{#1}
			\end{minipage} &
			\noindent \normalfont \small #2
		\end{tabularx}}

\newcommand{\EducationEntry}[5]{
		\large \noindent \textbf{#1} \hfill
		\Gbox{\small \color{clr_darkgrey}#3} \par
		\normalsize \noindent \textit{#2} \hfill	  % School
		\noindent \textit{#4} \par        % School
		\noindent\hangindent=2em\hangafter=0 \small #5 % Description
		\normalsize \par
		\sepspace}

\newcommand{\WorkEntry}[4]{
		\noindent \large \textbf{#1} \small \textit{\color{clr_darkgrey} @ #3} \hfill
		\Gbox{\small \color{clr_darkgrey}#2} \par
		\noindent \hangindent=0em\hangafter=0 \small #4
		\normalsize
		\sepspace}

\newcommand{\ProjectEntry}[4]{
		\noindent \large \textbf{\href{#2}{#1}} \hfill
		\Gbox{\small \color{clr_darkgrey}#3} \par
		\noindent \hangindent=0em\hangafter=0 \small #4
		\normalsize
		\sepspace}

%%% ------------------------------------------------------------
%%% BEGIN DOCUMENT
%%% ------------------------------------------------------------

\begin{document}

\MyName{Benjamin Mastripolito}{benpm@cs.utah.edu \(\star\) \href{https://benpm.github.io}{benpm.github.io}}

\sepspace


%%% Education
%%% ------------------------------------------------------------
\NewPart{Education}

\EducationEntry
{MS Computing, Graphics \& Visualization}
{University of Utah}
{Aug 2022 - July 2024}
{3.9 GPA}

\EducationEntry
{BS Computer Science}
{New Mexico Institute of Mining and Technology (NMT)}
{Aug 2016 - May 2020}
{3.79 GPA}



%%% Work experience
%%% ------------------------------------------------------------
\NewPart{Professional Experience}

\WorkEntry{Graduate Research Assistant}
{Fall 2023 - Now}{University of Utah}{\noindent
	Research with Dr. Cem Yuksel (\textit{cem@cemyuksel.com}) on a novel, intuitive 3D modeling methodology combining workflows from polygonal modeling, CAD, and sculpting.
	\begin{itemize}
		\item Co-wrote a fully functional 3D modeling application in C++ and OpenGL as a preliminary for a future user study and publication
		\item Implemented a bespoke half-edge mesh representation scheme, reducing 36 to 16 bytes per triangle, minimizing GPU memory transfer overhead during modeling tasks such as extrusion and slicing, which were implemented using compute shaders
		\item Implemented mesh editing tools such as extrusion, scale, translate, rotate, manipulation of geometry using harmonic weight calculation to preserve mesh details
	\end{itemize}
}

\WorkEntry{Post-Baccalaureate Student}
{June 2020 - July 2022}{Los Alamos National Laboratory}{\noindent
	\begin{itemize}
		\item Researched parallel interpolation algorithms for large scientific data sets under the mentorship of Dr. Daniel Sheppard (\textit{danielsheppard@lanl.gov}), publishing \href{https://asmedigitalcollection.asme.org/computingengineering/article-abstract/22/2/021009/1121736/simd-Optimized-Search-Over-Sorted-Data}{"SIMD-Optimized Search Over Sorted Data"} in ASME Computing and Engineering. The result was a SIMD-optimized binning and search algorithm that improved over binary search by 110\% in vectorized contexts.
	\end{itemize}
}

\WorkEntry{Undergraduate Research Assistant}
{Jan 2019 - June 2020}{New Mexico Institute of Mining and Technology}{\noindent
	\begin{itemize}
		\item Worked with Dr. Denis Cohen (\textit{denis.cohen@gmail.com}), and a fellow student to rewrite a complex 3D landslide simulation from scratch, while redesigning it to operate in parallel on unstructured meshes rather than a regular grid. We reduced the runtime for large datasets by 80\% while eliminating nearly 50\% of the code.
	\end{itemize}
}

\WorkEntry{Parallel Computing Research Internship}
{June 2019 - Aug 2019}{Los Alamos National Laboratory}{\noindent
	\begin{itemize}
		\item Worked as a student in the Parallel Computing Summer Research Internship at LANL researching performance analysis techniques on parallel algorithms, under mentorship of Dr. Rao Garimella (\textit{rao@lanl.gov})
	\end{itemize}
}


%%% Projects
%%% ------------------------------------------------------------
\NewPart{Projects}

\ProjectEntry{Elastic Hair Simulation and Rendering}
{https://github.com/alpers-git/StrandStorm}{2023}{
	\begin{itemize}
		\item Implemented an interactive hair simulation using discrete elastic rods and rigid-body collisions using position-based dynamics. Wrote an OpenGL compute shader able to generate 64 rendered hairs per simulated hair in < 10ms every frame.
		\item Implemented deep opacity shadow maps, alongside a physically-based hair shading model for a realistic-appearing result.
	\end{itemize}
}

\ProjectEntry{GPU-Accelerated Surface Meshing}{https://github.com/benpm/gl_playground}{2023-2024}{
	Implemented a dual-mesh isosurface extraction algorithm as well as a surface geometry smoothing scheme using a trilinear interpolation as a multi-stage OpenGL compute shader pipeline. The pipeline is able to generate 0.7 million vertices and triangles in less than 10 milliseconds on modern hardware.
}

\ProjectEntry{WebGL Compact Cellular Automata}{https://medium.com/better-programming/multi-state-cellular-automata-in-webgl-2bff79bf08fb}{2020}{
	\begin{itemize}
		\item Developed a method for storage of arbitrary, multi-state cellular automata rules which exponentially improves on naive implementations that rely on enumerating permutations of neighbor sequences.
		\item The method is able to store and retrieve rules from 4k textures, with a maximum of 14 states fitting into less than \(2^{25}\) bits. A naive implementation would require nearly \(2^{44}\) bits. Outside the constraints of a 4k texture, the method generalizes to any number of states, and any size or dimension of neighborhood, not just the standard 8-cell 2D moore neighborhood.
	\end{itemize}
}

\ProjectEntry{Fast Particle Simulation}
{https://github.com/TheFutureGadgetsLab/petri}{2021}{
	Co-developed a highly optimized parallel particle simulator in Rust, able to simulate and render over 1,000,000 colliding particles at interactive frame rates.
}

\ProjectEntry{CUDA Raytracer}
{https://github.com/benpm/cuda-raytracer}{2020}{
	Created a parallel raytracing program using NVIDIA CUDA C++ and OpenGL. Supports BSDF materials.
}


%%% Skills
%%% ------------------------------------------------------------
\NewPart{Technical Skills}

Confident with C++, OpenGL, and CMake. Somewhat experienced with CUDA, DirectX 12, OpenMP, Rust, WebGPU, WebGL, and Vulkan. Working knowledge of linear algebra, numerical methods, and parallel algorithm design.

\end{document}
